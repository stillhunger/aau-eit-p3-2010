\documentclass{beamer}
\usetheme{Goettingen}
\usepackage[danish]{babel}
\usepackage[utf8]{inputenc}
\usepackage{times}
\usepackage{pstricks}
\usepackage{xmpmulti}
\usepackage{multimedia}
\usepackage{amsmath,amssymb}

\begin{document}

%%%%%%%%%%%%%%%%%%%%%%%%%%%%%% Kravspec %%%%%%%%%%%%%%%%%%%%%%%%%%%%%%

\section{Forforstærker}
\begin{frame}{formål}
\begin{itemize}
\item Forstærke signal fra mic op til linjeniveau
\item Mic out: 0,8-200 mV
\item Opnåes: 200 mV - 2 V
\item Forholdet mellem høj og lav ikke ens
\item Dermed valgt out for alm tale tættere på som midt
\item 60 - 80 dB(A)
\end{itemize}
\end{frame}

%%%%%%%%%%%%%%%%%%%%%%%%%%%%%% Løsningsdesign %%%%%%%%%%%%%%%%%%%%%%%%%%%%%%

\begin{frame}{Krav}
\begin{itemize}
\item Indgangsimpedans bestemt ud fra tommelfingerregel
\item Forstærkningen er beregnet ud fra de fundne spændingsniveauer 
\end{itemize}
\end{frame}

%%%%%%%%%%%%%%%%%%%%%%%%%%%%%% Samlet accepttest %%%%%%%%%%%%%%%%%%%%%%%%%%%%%%


\begin{frame}{Opbygning}
\begin{itemize}
\item Op-amp foreslag - pga. så meget muligt diskret
\item To common-emitter med uafkoblet
\item I modsætning til common base og collector er spændingsforstærkning næsten ikke afhængig af trans parametrer
\item Dog stor udgangsimpedans, men den beregnes belastet
\item Designes uden AC-koblet emittermodstand til maks forstærkning 
\item Tilbagekobles gennem emittermodstand
\item trin har endelig forstærkning
\end{itemize}
\end{frame}


%%%%%%%%%%%%%%%%%%%%%%%%%%%%%% Konklusion %%%%%%%%%%%%%%%%%%%%%%%%%%%%%%

\begin{frame}{Simulering amplitude}
\begin{itemize}
\item Der ønskes 69,7 gange = 36,9 dB
\item Simuleret til 35 dB
\item For at rette op på det justeres emittermodstanden
\item Implementering: Indsættes pot metre
\end{itemize}
\end{frame}

\begin{frame}{Simulering korrigeret}
\begin{itemize}
\item Efter korrektion opnåes korrekt forstærkning
\item Maksimal dæmpning på 0,2 dB relativ 1 kHz
\item THD: 0.2 \% sammenlign med 0,5
\end{itemize}
\end{frame}

\begin{frame}{Accepttest amplitude}
\begin{itemize}
\item Indgangsimpedansen måltes til 22,1 k, hvor afvigelsen forklares med 1 \% modstande
\item Forstærkningen er indstillet og dermed korrekt
\end{itemize}
\end{frame}

\begin{frame}{Accepttest 20-63 Hz}
\begin{itemize}
\item Forskel 0,2 dB
\item Krav: Under 0,75 dB
\end{itemize}
\end{frame}

\begin{frame}{Accepttest 12 kHz-20 kHz}
\begin{itemize}
\item Forskel 0,8 dB
\item Krav: Under 0,75 dB
\item Fejlen tilskrives testudstyrets kapacitive belastning
\end{itemize}
\end{frame}

\begin{frame}{Simuleret 12 kHz-20 kHz fejl}
\begin{itemize}
\item Før simuleres den til 0,2 dB
\item Med belastning: 0,6 dB
\item Kapacitiv belastning bidrager med 0,4 dB
\end{itemize}
\end{frame}

\begin{frame}{Accepttest THD}
\begin{itemize}
\item Målt ved maks udsving: Maksimal 0,22 \%
\item Krav: Under 0,5
\item simuleret 0,2
\end{itemize}
\end{frame}

\end{document}