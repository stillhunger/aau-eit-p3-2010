\section{Forforstærker}

\begin{frame}{Forforstærker}
\begin{itemize}
\item Forstærke signal fra mikrofon
\item Mikrofon: MCE-4000
\item Mikrofons output spænding: 0,8 - 200 mV
\item Output som skal opnås: 200 mV - 2 V
\item Lineær forstærkning ikke muligt
\item Valgte peakspændinger beregnet ud fra forventeligt lydtryk
\end{itemize}
\end{frame}

\begin{frame}{Forforstærker - krav}
\begin{itemize}
\item Indgangsimpedans: 22 k\ohm
\item Frekvensgang: 
\begin{itemize}
\item  under 0,375 dB ved 20 Hz - 20 kHz, ref. 1 kHz 
\item  under 0,75 dB fra 20 Hz til 63 Hz 
\item  under 0,75 dB fra 12,5 kHz til 20 kHz
\end{itemize}
\item Forvrængning: < 0,5 \%
\item Forstærkning: 69,7 gange ved 22k\ohm indgangsimpedans og ved 1 kHz
\end{itemize}
\end{frame}

\begin{frame}{Forforstærker - opbygning}
\begin{itemize}
\item To common-emitter forstærkere med uafkoblet emittermodstand
\item Der vælges to trin for at opnå en stor mængde tilbagekobling
\item Trin 1: 10 ganges forstærkning
\item Trin 2: 6,97 ganges forstærkning
\item Hvorfor rækkefølgen??!
\item Skal vi have et billede af kredsløbet?
\end{itemize}
\end{frame}

\begin{frame}{Forforstærker - Accepttest}
\begin{itemize}
\item 
\item Mikrofon: MCE-4000
\item Mikrofons output spænding: 0,8 - 200 mV
\item Output som skal opnås: 200 mV - 2 V
\item Lineær forstærkning ikke muligt
\item Valgte peakspændinger beregnet ud fra forventeligt lydtryk
\end{itemize}
\end{frame}