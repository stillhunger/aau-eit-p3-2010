\chapter{Appendiks C}
\label{beregning-af-analog-attenuator}
\section*{Beregning af analog attenuator}

Den analoge attenuator der benyttes i volumenkontrollen består af to attenuatorer, adskilt af en buffer. Den første dæmper i seks trin med 8 dB mellem hvert trin. Den anden dæmper i syv trin med 1 dB mellem hvert trin. Bregningerne er opstillet i ligningerne \ref{beregning-af-analog-attenuator-ligning1} til \ref{beregning-af-analog-attenuator-ligning5}

\begin{equation}
\label{beregning-af-analog-attenuator-ligning1}
\frac{R_6}{R_1 + R_2 + R_3 + R_4 + R_5 + R_6} = 10^{\frac{-40}{20}}
\end{equation}

\begin{equation}
\frac{R_6 + R_5}{R_1 + R_2 + R_3 + R_4 + R_5 + R_6} = 10^{\frac{-32}{20}}
\end{equation}

\begin{equation}
\frac{R_6 + R_5 + R_4}{R_1 + R_2 + R_3 + R_4 + R_5 + R_6} = 10^{\frac{-24}{20}}
\end{equation}

\begin{equation}
\frac{R_6 + R_5 + R_4 + R_3}{R_1 + R_2 + R_3 + R_4 + R_5 + R_6} = 10^{\frac{-16}{20}}
\end{equation}

\begin{equation}
\label{beregning-af-analog-attenuator-ligning5}
\frac{R_6 + R_5 + R_4 + R_3 + R_2}{R_1 + R_2 + R_3 + R_4 + R_5 + R_6} = 10^{\frac{-8}{20}}
\end{equation}

Da der er fem ligninger med seks ubenkendte bestemmes $R_6$ til 10 k\ohm, det resultere i løsningen i ligning \ref{beregning-af-analog-attenuator-resultat1}

\begin{equation}
\label{beregning-af-analog-attenuator-resultat1}
R_1 = 647 k\ohm, R_2 = 194 k\ohm, R_3 = 95,3 k\ohm, R_4 = 37,9 k\ohm, R_5 = 15,1  k\ohm, R_6 = 10,0 k\ohm
\end{equation}