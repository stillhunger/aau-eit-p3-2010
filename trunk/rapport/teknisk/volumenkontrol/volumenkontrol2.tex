\subsection{VCO}
\label{volumenkontrol-simulering-vco}

En VCO, Voltage Controlled Oscillator, levere et konstant signal hvor frekvensen er afhængig af en kontrolspænding. Der er taget udgangspunkt i en VCO fra databladet for en LM324. VCO'en kan deles op i to blokke én integrator og én schmidt-trigger. Det er udgangen fra schmidt-triggeren der bestemmer hvilken af de to spændinger integrator arbejder udfra. Triggerspændingerne på schmidt-triggeren er givet ved ligning \ref{equ:vco1} og \ref{equ:vco2}.

\begin{equation}
\label{equ:vco1}
V_L = \mathrm{\frac{\frac{1}{\frac{1}{40~k\ohm} + \frac{1}{40~k\ohm}}}{40~k\ohm + \frac{1}{\frac{1}{40~k\ohm} + \frac{1}{40~k\ohm}}} \cdot 5~V = 1,67~V}
\end{equation}

\begin{equation}
\label{equ:vco2}
V_U = \mathrm{\frac{40~k\ohm}{\frac{1}{\frac{1}{40~k\ohm} + \frac{1}{40~k\ohm}} + 40~k\ohm} \cdot 5~V = 3,33~V}
\end{equation}

Spændingen angivet i ligning \ref{equ:vco2} er dog en tilnærmelse, da det ikke vil være muligt for operationsforstærkeren at levere forsyningsspænding som udgangsspænding.