\subsection*{VCO}
\label{volumenkontrol-vco}

En VCO, Voltage Controlled Oscillator, leverer et konstant signal hvor frekvensen er afhængig af en kontrolspænding. Kontrolspændingen er spændingen over $C_2$, ladekondensatoren, minus én diodespænding. Der er taget udgangspunkt i en VCO fra databladet for en LM324. VCO'en kan deles op i to blokke; én integrator og én schmidt-trigger. Det er udgangen fra schmidt-triggeren der bestemmer hvilken af de to spændinger integrator arbejder udfra. Triggerspændingerne på schmidt-triggeren er givet ved ligning (\ref{equ:vco1}) og (\ref{equ:vco2}). 

\begin{equation}
\label{equ:vco1}
V_L = \mathrm{\frac{\frac{1}{\frac{1}{40~k\ohm} + \frac{1}{40~k\ohm}}}{40~k\ohm + \frac{1}{\frac{1}{40~k\ohm} + \frac{1}{40~k\ohm}}} \cdot 5~V = 1,67~V}
\end{equation}

\begin{equation}
\label{equ:vco2}
V_U = \mathrm{\frac{40~k\ohm}{\frac{1}{\frac{1}{40~k\ohm} + \frac{1}{40~k\ohm}} + 40~k\ohm} \cdot 5~V = 3,33~V}
\end{equation}

Spændingen angivet i ligning (\ref{equ:vco2}) er dog en tilnærmelse, da det ikke vil være muligt for operationsforstærkeren at levere forsyningsspænding som udgangsspænding.

Frekvensen VCO'en vil svinge med, er givet ved ligning (\ref{equ:vco3}), udtrykket er udledt i appendiks A??.

\begin{equation}
\label{equ:vco3}
f = \frac{1}{\frac{2 \cdot V_{CC} \cdot C \cdot R_1^2}{3 \cdot V_C \cdot (R_1 - R_4)}} = \frac{3 \cdot V_C \cdot (R_1 - R_4)}{2 \cdot V_{CC} \cdot C \cdot (R_1)^2}
\end{equation}

En approximering af forholdet mellem høj og lav i duty-cyclen, er givet ved forholdet mellem $R_1$ og $R_4$, i dette tilfælde $\frac{R_4}{R_1} = \frac{100~k\ohm}{1000~k\ohm}=0.1$. Dette skyldes at det er disse to modstande $C_1$ op- og aflades igennem.  Grunden til at frekvensen stiger når spændingen stiger, er at opampen altid vil presse sine indgange til at være ens. Da der på plus-indgangen sidder en spændingsdeling, som giver halvdelen af styringsspændingen, vil der ligge det samme på minus-benet. Dette betyder, at spændingsfaldet over $R_1$ altid vil være halvdelen af styringsspændingen. Dette vil betyde at der vil løbe en strøm igennem $R_1$ ind i kondensatoren. Når transistoren leder, vil den lede strømmen, som løber igennem $R_1$ samt den strøm der kommer fra kondensatoren. Når kondensatoren aflader igennem transistoren vil den prøve at trække minus benet ned. Dette prøver op-ampen at undgå ved at øge output spændingen. Hvis man lader denne proces fortsætte uendeligt vil plus- og minus-benet være ens, indtil op-ampen rammer sin maksimale spænding. Herefter vil den ikke være i stand til at regulere spændingen på minus-benet, hvilket vil resultere i at spændingen på minus-benet vil være spændingsdelingen mellem $R_1$ og $R_4$. Dette forhindrer Schmidt-triggeren dog, ved at ændre på hvor strømmen igennem $R_1$ har mulighed for at løbe hen. Når der løber strøm til minus-benet vil dette føre til en spændingstigning. Da op-ampen stadig vil forsøge at holde indgangene éns, vil dette betyde et spændingsfald på outputtet. Det er denne effekt der gør svingningen mulig.

Outputtet fra Schmidt-triggeren er højt, som standard, da outputtet fra integratoren, når denne ikke har en høj nok styringsspænding til at gå i gang, vil være lavt. Dette betyder at AND-gaten der giver signalet videre til tælleren kan give et positivt output, når en knap trykkes ned en enkelt gang. På denne måde vil det være muligt benytte knapperne til at regulere et enkelt niveau op eller ned, samtidig med muligheden for at holde dem inde, og aktivere VCO'en, for at regulere volumeniveauet hurtigere.


%Denne strøm kan kun løbe ind i kondensatoren, hvilket vil oplade denne. Når kondensatoren aflades gennem $R_4$ vil minusbenet gå mod en lavere spænding. For at undgå dette vil op-ampen outputte en højere spænding, for at forsøge at holde begge input på samme spændingsniveau. Kondensatoren blev ved med at aflade, ville op-ampen til sidst nå sit maksimale output. Når dette sker, vil minus benet falde til en spændingsdeling mellem $R_1$ og $R_4$, hvis transistoren $Q_1$ ses som en kortslutning. Dette vil dog ikke ske i praksis, da schmidt-triggeren forhindrer netop dette. 


%I takt med at spændingen stiger over ladekondensatoren vil spændingen over $R_1$ og $R_2$ stige. Dette vil betyde at strømmen ind i kondensatoren, $C_1$ bliver større. Dette forklarer hvordan low-tiden bliver mindre. Kondensatoren vil stadig skulle aflade igennem $R_4$, hvilket umiddelbart ikke lægger op til en kortere høj-tid. Dog vil spændingen over $R_2$ også stige, hvilket gør outputtet fra integratoren højere. Dette vil betyde en lavere spænding over kondensatoren, hvilket vil betyde at den ikke kan lade lige så meget op og den derfor hurtigere kan aflades.
%Hvis V_C stiger vil spændingen på V_+ stige og dermed også V_-. Det vil lave en størrer spænding over R_1 og R_4, og strømmen igennem dem vil så også stige. Hvis kondensatorens kapacitet er konstant og strømmen den op- og aflades med er stigende vil op- og afladetiden falde.