\section{Design}

\begin{figure}[h]
\centering
\includegraphics[scale=0.4]{teknisk/indgangsvaelger/signal-taend-sluk.png}
\caption{Opbygning af indgangsvælgeren}
\label{indgangsvaelger-overordnet}
\end{figure}
For at slukke signalet, er der valgt at benytte en transistor, til at trække signalet til stel. Da vores spændingssving kan gå ned til 0,2~V, er der mulighed for at der vil løbe nogle små strømme. Derfor vælges en FET-transistor, da denne, modsat en BiPolær transistor, er linæer ved meget små strømme. Optimalt ville en transistor uden reverse-diode være foretrække, da dette vil tillade at source-spændingen er lavere end drainspændingen, som vil være tilfældet ved et AC-signal omkring 0~V. Da det ikke var muligt at fremskaffe en sådan, benyttes i stedet et DC-offset, for at sørge for at DC-offset $-$ AC-Peakværdi $>$ 0 når transistoren er slukket, hvilket umuliggører at der kan løbe en reverse strøm i transistoren.

Modstandene $R_1$ og $R_2$ er begge valgt til 100~k\ohm, for at give et DC-offset på ca 2,5~V, halvdelen af V1. Dette gælder dog kun når transistoren er slukket. I det transistoren tændes, sættes $R_2$ parallelt med $R_3$, hvilket trækker DC-offsettet længere ned.
%\begin{equation}
%5V\cdot \frac{\frac{1}{\frac{1}{R_2}+\frac{1}{R_3}}}{R_1+\frac{1}{\frac{1}{R_2}+\frac{1}{R_3}}}=1.11 V
%\end{equation}
I dette tilfælde er, hvor transistoren skal tage hele signalet, er DC-offsettet dog underordnet. 
Modstanden $R_3$ er valgt ud fra, at signalet skal se en indgangsimpedans på minimum 22 k\ohm. Når transistoren er tændt er indgangsimpedansen mindst. $R_1$, $R_2$ og $R_4$ sidder så alle parallelt, hvilket giver ligning \ref{eq:indgangr4udregning} og \ref{eq:indgangr4}:

\begin{equation}
\label{eq:indgangr4udregning}
\frac{1}{\frac{1}{R_1}+\frac{1}{R_2}+\frac{1}{R_4}}=22~\mathrm{k}\ohm
\end{equation}

\begin{equation}
\label{eq:indgangr4}
R_4=39,29~\mathrm{k}\ohm
\end{equation}

$R_4$ er så valgt til 40,2~k\ohm, for at passe med E96-rækken.
For at kunne afbryde de enkelte signaler, kan transistoren $U_2$ trække signalet til stel. for at tillade at hele signalet bliver trukket til stel, skal hele den strøm der løber igennem systemet føres ned igennem transistoren. På en mosfet-transistor, som benyttes i dette kredsløb, løber der ingen strøm ind i basis, den er udelukkende afhængig af basis-emitter spændingen. Ved at påtrykke en spænding på 5V, som er outputtet fra de gates vi bruger, er det muligt at tillade signalet at løbe igennem transistoren.
Modstanden $R_5$ har ikke nogen indvirkning på indgangsimpedansen når transistoren er tændt og signalet derfor er slukket. Når signalet er tændt, sidder den i serie med $R_4$, hvilket giver en højere indgangsimpedans. For at summationsforstærkeren efter, skal kunne fungere med en forstærkning på 1, skal tilbagekoblingsmodstanden over transistoren være lig med den modstand signalet ser, på vej til forstærkeren. Denne er afhængig af $R_4$ og $R_5$, som tilsammen skal give $R_m$. Med $R_7$ defineret til 80,4 k\ohm sættes $R_5$ til 40,2 k\ohm . Indgangsimpedansen kan så udregnes, som i ligning \ref{eq:indgangrindgang}.

%Den er valgt til 40.2 k\ohm, det samme som $R_3$, for at gøre produceringen enklere. Ved at bruge modstande med de samme værdier sænkes antallet af benyttede komponenter, hvilket letter indkøbs- samt produceringsomkostninger. Når signalet er tændt, sidder den i serie med $R_3$, hvilket giver en højere indgangsimpedans:

\begin{equation}
\label{eq:indgangrindgang}
R_{\mathrm{indgang}}=\frac{1}{\frac{1}{R_1}+\frac{1}{R_2}+\frac{1}{R_3+R_4}}=30,8~\mathrm{k}\ohm
\end{equation}

Efter at have opstillet de forskellige modstandsværdier, er det muligt at udregne værdien af afkoblingskondensatorerne i kredsløbet. Disse kan udregnes som en spændingsdeling mellem en seriekoblet modstand og kondensator, seriekoblet med en modstand, som vist på figur \ref{crvd}. I dette tilfælde vil $R_U$ være udgangsimpedansen på det foregående led, og $R_I$ være indgangsimpedansen på det efterfølgende. Impedansen i en kondensator, i frekvensdomænet er $\frac{1}{s\cdot C}$. Dette kan opstilles i følge spændingsdelingsformel som ligning \ref{eq:indganghoejpas}.

\begin{equation}
\label{eq:indganghoejpas}
\frac{V_{\mathrm{out}}}{V_{\mathrm{in}}}=\frac{R_I}{R_I+(R_U+\frac{1}{s\cdot C})}
\end{equation}

For at få en dæmpning på 3 dB, som er den ønskede dæmpning i knækpunktet, skal $\frac{V_{\mathrm{out}}}{V_{\mathrm{in}}}=10^{\frac{-3}{20}}\approx0,7$. 
Dette giver 2 ubekendte, $s$ og $C$. LaPlace variablen $s$ kan ses som $2\cdot \pi \cdot f$, hvor $f$ er den ønskede frekvens ved knækket, som vist i ligning \ref{eq:indgang2pif}. Der opstilles et udtryk for $C$ i ligning \ref{eq:indgangcudregning}.

\begin{equation}
\label{eq:indgang2pif}
10^{\frac{-3}{20}}=\frac{R_I}{R_I+(R_U+\frac{1}{2\cdot\pi\cdot 2\cdot C})}
\end{equation}

\begin{equation}
\label{eq:indgangcudregning}
C=\frac{-1}{2}\cdot{10^{\frac{-3}{20}}}{\pi\cdot f\cdot(10^{\frac{-3}{20}}\cdot R_I+\frac{-3}{20}}\cdot R_U - R_I)
\end{equation}

Frekvensen f bestemmes til 2 Hz, én dekade før den ønskede, for at opnå en lav dæmpning ved de ønskede 20 Hz. Dæmpningen ved 20 Hz kan så udregnes. Formlen for et standard højpas-filter opstilles i ligning \ref{eq:indgangstandardhoejpas}. Denne omskrives så til $j\cdot\omega$ notation i ligning \ref{eq:indgangjomega}.

\begin{equation}
\label{eq:indgangstandardhoejpas}
%\tau = R\cdot C
H(s)=\frac{s\cdot\tau}{1+s\cdot\tau}
\end{equation}

\begin{equation}
\label{eq:indgangjomega}
|H(j\omega)|=\frac{\omega\cdot\tau}{\sqrt{1+(\omega\cdot\tau)^2)}}=\frac{1}{\sqrt{\frac{1}{(\omega\cdot\tau)^2}+1}}
\end{equation}

Som i følge dekadereglen, sættes $\omega\tau = 10$, som i ligning \ref{eq:indgangjomega10}.

\begin{equation}
\label{eq:indgangjomega10}
\frac{1}{\sqrt{\frac{1}{100}+1}}=\frac{1}{\sqrt{1,01}}\approx -0,043~\mathrm{dB}
\end{equation}

Ud fra dette kan de forskellige værdier for $C$ udregnes, afhængigt af de impedanser de ser ind i.
Indgangsimpedansen for $C_1$ er fundet til 30,8~k\ohm . Indtastes dette i ovennævnte formel findes værdien for $C_1$ til mindst 8~µF. Alt under dette vil give en højere knækfrekvens, hvilket ikke er at ønske. Alt højere vil dog give en større indsvingningstid, hvilket, i forhold til en højere knækfrekvens, er at foretrække, dog heller ikke ønskeligt.
$R_3$ skal være lig med den samlede impedans den minus porten på op-ampen kigger ind i. Denne udgøres af en parallel kobling af hver indgang og tilbagekoblingsmodstanden $R_7$, som vist i ligning \ref{eq:indgangr3}

\begin{equation}
\label{eq:indgangr3}
R_3 = \frac{1}{\frac{1}{\frac{\frac{1}{\frac{1}{R_1}+\frac{1}{R_2}}+R_4+R_5}{3}}+\frac{1}{R_7}}
\end{equation}

Efter summationsforstærkeren sidder en spændingsdeler, som giver mulighed for vælge hvor stærkt signalet skal være. Denne benyttes til  at sørge for at lige gyldigt hvor mange af indgangene der er tændt for, vil signalet altid være mellem 0,2 og 2~V peak. Da de niveauer der er mulige at få enten er 6, 4 eller 2~V vil det, for at give 2~V output, være nødvendigt med en dæmpning på hhv. $\frac{1}{3}$, $\frac{1}{2}$ og 1. Dette muliggøres ved hjælp af en spændingsdeler med 3 modstande, som vist på figur \ref{outputdeler}. Efter denne sidder en buffer, for at impedansen der ses af afkoblingskondensatoren, som sidder mellem indgangsvælgeren og volumenkontrollen, vil være konstant.