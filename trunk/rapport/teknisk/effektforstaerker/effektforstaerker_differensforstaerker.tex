\subsection{Differensforstærker}
\label{effekt_differensforstaerker}

\begin{figure}[h]
\centering
\includegraphics[scale=.2]{teknisk/effektforstaerker/differensforstaerker.png}
\caption{Diagram over differensforstærkeren hvor strømspejl og konstantstrømsgenerat er markeret}
\label{fig:differensforstaerker}
\end{figure}

Differensforstærkerens formål er at danne platform for en tilbagekobling ved at forstærke forskellen mellem signalet på input1 og input2 samt undertrykke common-mode signaler. Strømgeneratoren sørger for at der løber en konstant strøm, $\frac{1}{2}I_\mathrm{bias}$, i de to grene af differensforstærkeren når input1 = input2. For at disse strømme effektivt skal være ens er det nødvendigt at benytte matchede transistorer til både strømspejlet og Q1 og Q2. Stiger strømmen gennem Q2 med $\Delta I$, som følge af en øget spænding på input2 relativt til input1, vil strømspejlet øge strømmen i den modsatte gren så strømmene i de to grene er ens. Da der nu løber $\frac{1}{2}I\mathrm{bias} + \Delta I$ i begge grene, men kun kan løbe $I_\mathrm{bias}$ gennem konstantstrømsgenerator vil der nødvendigvis løbe $I_\mathrm{bias} -\Delta I$ gennem Q1 og $2\Delta I$ ud i outputgrenen. Denne sammenhænge gælder både med en strømforøgelse og -formindskelse gennem Q2. På denne måde styres spændingsforstærkeren, som er dokumenteret i afsnit \ref{effekt_spaendingsforstaerker}. 

Biasstrømmen, $I_\mathrm{bias}$, som konstantstrømsgeneratoren skal generere vælges til 2 mA, da transistorparametrene for den anvendte transistor, BC547b (Q3), er veldefinerede ved denne collectorstrøm, hvilket letter beregningerne. Når der løber en strøm gennem generatoren på 2 mA vil der, hvis differensforstærkeren er i balance, løbe 1 mA i hver gren. Der antages at transistorparametrene angivet ved en collectorstrøm på 2 mA også er gældende for en strøm på 1 mA. 
Konstantstrømsgeneratoren designes ud fra samme procedure som benyttes i underafsnittet Konstantstrømsgenerator i afsnit \ref{effekt_stroemforstaerker}. 


Differensforstærkningen, $A_d$, i differensforstærkeren er et udtryk for hvor meget spændingsdifferensen mellem inputsignalet og det tilbagekoblede signal forstærkes. $A_d$ har indflydelse på CMRR 


\subsubsection*{Simulering}