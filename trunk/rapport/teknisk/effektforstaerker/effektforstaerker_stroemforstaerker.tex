\subsection*{Strømforstærker}
\label{effekt_stroemforstaerker}

\begin{figure}[h]
\centering
\includegraphics[scale=0.4]{teknisk/effektforstaerker/blokdiagram-stroemforstaerker.png}
\caption{Diagram over strømforstærkeren}
\label{fig:blokdiagram-stroem}
\end{figure}

Strømforstærkeren opbygges som vist på figur \ref{fig:blokdiagram-stroem}. Dog vil konstantstrømsgeneratoren blive bygget i diskret elektronik. Der er valgt at der benyttes en BDX33B og en BDX34B \fixme{kilde: BDX33-34.pdf} som udgangstransistorer. Dette er darlingtontransistorer, som er valgt da de har en $h_{\mathrm{FE}}$ på minimum 750 og kan klare en $I_C$ på op til 10 A. Desuden er de let tilgængelige til projektet.Som vist, i afsnit \ref{valg_kortslutningssikring}, skal der igennem $R_{\mathrm{load}}$ løbe en $I_{\mathrm{peak}}$ på 2,24 A for at opnå en udgangseffekt på 20 W. Dette betyder desuden, som det også er vist i afsnit \ref{valg_kortslutningssikring}, at der skal være en $V_{\mathrm{peak}}$ på 17,9 V over belastningen. Da de valgte darlingtontransistorer har en $V_{\mathrm{BE}}$ på op til 2,5 V, vælges forsyningsspændingen til $\pm$23 V, hvormed der også er plads til et spændingsfald over $R_E$ og en transistor i konstantstrømsgeneratoren.\\\\
Størrelsen af $R_E$ bestemmes med udgangspunkt i at den skal skabe termisk stabilitet og til bestemmelsen af denne startes der derfor med at kigge på et termisk ekvivalentdiagram for de valgte darlingtontransistorer og de tilgængelige køleplader\fixme{kilde: koeleplade.pdf}. 

\begin{figure}[h]
\centering
\includegraphics[scale=0.2]{teknisk/effektforstaerker/termisk_ekvivalentdiagram.png}
\caption{Termisk ekvivalentdiagram for udgangstransistorerne}
\label{fig:term-dia}
\end{figure}

På figur \ref{fig:term-dia} ses ekvivalentdiagrammet, hvor; temperatur er spænding, effekt er strøm og termisk modstand er modstand. Flere af komponenterne har samme benævnelser, da de også antager samme værdier og da der under udregningerne dermed ikke vil blive set på dem enkeltvis.\\
Størrelsen af $P_D$ er effekten afsat i en enkelt udgangstransistor og er givet ved udregningen i formel (\ref{equ:pd})\fixme{kilde: Jan Mikkelsen, mm19}. Denne formel er ganske vist givet for et klasse B udgangstrin, men den er også gældende for et klasse AB udgangstrin, da hvilestrømmem i et klasse AB udgangstrin kun er betydeligt forskellig fra nul i perioden, hvor der er et skifte i hvilken transistor, som er aktiv \fixme{kilde: klasse-ab.pdf}.

\begin{equation}
\label{equ:pd}
P_D = \frac{1}{\pi^2} \cdot \frac{(V_{CC})^2}{R_{\mathrm{load}}} = \frac{1}{\pi^2} \cdot \frac{(23~\mathrm{V})^2}{8~\ohm} = 6,7~\mathrm{W}
\end{equation}

Fra darlingtontransistorernes datablad haves $R_{\mathrm{jc}} = 1,78~\tfrac{\celsius}{\mathrm{W}}$ og $T_{\mathrm{j,max}} = 150~\celsius$. Databladet for kølepladen giver $R_{\mathrm{cs}} = 1,4~\tfrac{\celsius}{\mathrm{W}}$ når der anvendes isoleringspasta. Fra DIN45500 \fixme{kilde: DIN45500.pdf} fåes at HiFi-forstærkeren skal kunne holde til en omgivelsestemperatur på 35~\celsius, hvilket er $T_a$ i ekvivalentet. Udfra dette opstilles, ved simpel kredsløbsteori på ekvivalentkredsløbet på figur \ref{fig:term-dia}, formel (\ref{equ:tjmax}) til beregning af $R_{\mathrm{sa}}$ ved $T_j$  på sin maksimale værdi. 

\begin{equation}
\label{equ:tjmax}
T_j = T_a + P_D \cdot (R_{\mathrm{jc}} + R_{\mathrm{cs}} + 2 \cdot R_{\mathrm{sa}})
\end{equation}

Desuden opstilles udfra sikkerhedshensyn\fixme{Find gerne kilde der siger dette i stedet, hvis der er tid} et krav om at kølepladen maksimalt på blive 40 \celsius, hvorved bestemmelse af $R_{\mathrm{sa}}$ også kan foregå ved brug af formel (\ref{equ:smax}). 

\begin{equation}
\label{equ:smax}
40~\celsius = 2 \cdot P_D \cdot R_{\mathrm{sa}}
\end{equation}

Det ses at formel (\ref{equ:smax}) bliver den afgørende betingelse og at $R_{\mathrm{sa}}$ maksimalt må være $2,99~\tfrac{\celsius}{\mathrm{W}}$. I databladet for kølepladen ses at en $R_{\mathrm{sa}}$ på $2,9~\tfrac{\celsius}{\mathrm{W}}$ kan opnåes ved en køleplade på 110 mm, hvilket derfor vælges. Størrelsen af $R_E$ kan nu bestemmes ved formel (\ref{equ:rebestem})\fixme{kilde: Jan Mikkelsen, mm19 - med et lille twist}, hvor $K = - 2~\tfrac{\mathrm{mV}}{\celsius}$, $V_{CC} = 23~V$, $V_T = 26~\mathrm{mV}$ og $I_C = 2,24~A$.

\begin{equation}
\label{equ:rebestem}
R_E = - 2 \cdot K \cdot V_{CC} \cdot (R_{\mathrm{jc}} + R_{\mathrm{cs}} + R_{\mathrm{sa}}) - \frac{2 \cdot V_T}{I_C} = 536~m\ohm
\end{equation}

På figur \ref{fig:term-runaway} er vist en graf over effekten som bliver afsat i en af udgangstransistorerne, når der på dennes emitter sidder den beregnede $R_E$. Den lige grønne linie viser hvilken effekt den samlede køling er i stand til at lede væk, hvormed systemet er termisk stabilt så længe effektkurven for transistoren ligger under denne. Havde systemet ikke indeholdt en $R_E$, eller havde denne været for lille, ville effektkurven for transistoren ligge over den lige grønne linie, hvormed systemet ville have været termisk ustabilt. 

\begin{figure}[h]
\centering
\includegraphics[width=\textwidth]{teknisk/effektforstaerker/term-runaway.png}
\caption{Graf over termisk stabilitet med $R_E$}
\label{fig:term-runaway}
\end{figure}

Til at lave opstillingen på figur \ref{fig:term-runaway} er der antaget en hvilestrøm gennem transistoren på 3 mA.


\subsubsection*{Konstantstrømsgenerator}
\label{effektforstaerker-konstantstroemsgenerator1}

Konstantstrømsgeneratorens opgave er at levere en konstant strøm, denne strøm bruges til $V_{be}$-Multiplieren og til spændingsforstærkeren. I afsnit \ref{??} er det beregnet at konstantstrømsgeneratoren skal leveres 6 mA. Opbygningen der er valgt til konstansstrømsgereratoren er afbildet på figur \ref{konstantstroemsgenerator_model}
\begin{figure}[h]
\centering
\includegraphics[scale=0.35]{teknisk/effektforstaerker/stoemgenerator.png}
\caption{Diagram der viser opbygningen af konstantstrømsgeneratoren.}
\label{konstantstroemsgenerator_model}
\end{figure}
\newline
\newline
Til konstantstrømsgeneratoren er der valgt at anvende en BC547B som transistor og en 1N4148 som diode. BC547B har en $V_{be}$ spænding på maksimum 720 mV \fixme{Kilde: bc547 datablad}. Deraf designes det således at der er et spændingsfald på 720 mV over $D_1$, $D_2$, $R_{2}$ og $Q_{1_{BE}}$. I databladet for 1N4148\fixme{Kilde: 1N4148 datablad} fremgår det at den  ved en $I_F$ strøm på 8 mA har en $V_D$ spændingen på 720 mV. Strømmen igennem dioderne er givet ved hvor stor en strøm der løber igennem modstanden $R_{1}$. $R_{1}$ er således givet ved ligning (\ref{equ:stroemgenerator_effektforstaerker1}).

\begin{equation}
\label{equ:stroemgenerator_effektforstaerker1}
R_{1} = \frac{V_{CC} - 2 \cdot V_D}{I_F} = \mathrm{\frac{23~V - 2 \cdot 0,72~V}{8~mA} = 2,7~k\ohm}
\end{equation}

Da der nu ligger en konstant spænding over alle dioderne, kan man se dioden i transistoren som siddende parallelt, med det samme spændingsfald som $D_1$. Dette giver at der findes det samme, konstante, spændingsfald over $D_2$ og $R_{1}$, hvilket giver en konstant strøm gennem $R_{2}$. Derudfra kan $R_{2}$ bestemmes når der skal løbe 6 mA i den, ved ligning (\ref{equ:stroemgenerator_effektforstaerker2})

\begin{equation}
\label{equ:stroemgenerator_effektforstaerker2}
R_{2} = \frac{V_{D}}{I_{KONSTANT}} = \mathrm{\frac{720~mV}{6~mA} = 120\ohm}
\end{equation}

Hermed er Konstantstrømsgeneratoren designet til at levere 6 mA.
