\section{Design}
I dette projekt er der valgt så vidt muligt at designe alle løsninger med analog elektronik. Derfor er det valgt at forforstærkeren bygges af commonemittertrin med uafkoblet emittermodstand. Et commonemittertrins generelle opbygning er vist på figur \ref{fig:cekobling}.

\begin{figure}[h]
\centering
\includegraphics[scale=.6]{teknisk/forforstaerker/ceopkobling.png}
\caption{Generel form på commonemitterkobling med uafkoblet emittermodstand}
\label{fig:cekobling}
\end{figure}


Argumentet for dette valg er at det er det eneste trin, blandt commonemitter, -base og -collector, hvis spændingsforstærkning ikke afhænger af transistorparametre. Da transistorparametre blandt andet er afhængige af den anvendte transistors temperatur er det en betragtelig styrke ikke at skulle tage højde for dem. Spændingsforstærkningen i commonemittertrinnet er dog kun uafhængig af transistorparametre så længe følgende er gældende: $r_o >>R_C \| R_L$ og $i_e \approx i_c$.
Disse antagelser vil være gældende gennem hele designet af forforstærkeren. 
Forforstærkeren bygges af to commonemittertrin for at opnå den ønskede forstærkning på 63,3 gange. Dette skyldes at forstærkningen er givet ved ligning \ref{eq:gmbevis}.

\begin{equation}
A_v =  \frac{-gm \cdot R'_L}{1+gm \cdot R'_e} \approx  -\frac{R'_L}{R'_e} \Biggr\vert _{\frac{1}{gm}<R'_e}
\label{eq:gmbevis}
\end{equation}

Hvor $R'_e = R_e || R_E$ og $R'_L = R_L||R_C$. Det vil sige at jo tættere $R_e$ kommer på $\frac{1}{gm}$ jo mere indflydelse vil denne have på forstærkningen.

Det første trin skal have en forstærkning på 10 gange og det andet på 6,33 for at opnå den ønskede forstærkning, som vist på figur \ref{blok_forforstaerker}.

\begin{figure}[h]
\centering
\includegraphics[scale=.6]{teknisk/forforstaerker/blok_forforstaerker.png}
\caption{Blokdiagram over forforstærkerens byggeblokke samt lydsignalets vej}
\label{blok_forforstaerker}
\end{figure}

For at opnå så lav forvrængning som muligt designes hvert trin således at DC-forstærkningen er så stor som muligt for at have meget at tilbagekoble. At AC- og DC-forstærkningen kan være forskellige muliggøres af den AC-koblede $R_e$, da denne kun vil have indflydelse på AC-forstærkningen.

\subsection*{Design af færste trin}
Begge trin designes efter maksimal DC-forstærkning. DC-forstærkningen for et commonemittertrin er givet ved ligning \ref{eq:dcgain}.

\begin{equation}
|A_{v}|=\frac{1}{\left(\frac{V_t \cdot R_C}{V_{R_C}}+\frac{R'_S}{\beta}\right) \left(\frac{1}{R_C}+\frac{1}{R_L}\right)}
\label{eq:dcgain}
\end{equation}
Hvor $R'_S = R'_S||R_1||R_2$ og $V_t = 26 \cdot 10^{-3}$.

For at designe et kredsløb med maksimal DC-forstærkning justeres størrelsen af $R_C$ uden at variere spændingen over den, $V_{R_C}$. Den maksimale $R_C$ findes ved ligning \ref{eq:rcmaks}.

\begin{equation}
R_{\mathrm{C,maks}} = \sqrt{\frac{R'_S \cdot R_L \cdot V_{R_C}}{\beta \cdot V_t}}
\label{eq:rcmaks}
\end{equation}

I ligning \ref{eq:rcmaks} er $R'_S$ defineret som $R_1||R_2||R_S$. $R_S$ er fastlagt til 2,2 k\ohm, hvilket er mikrofonens udgangsimpedans\fixme{ref til MCE4000}. Parallelforbindelsen mellem $R_1$ og $R_2$ kan ikke beregnes men skal vælges. Indgangsimpedansen i kredsløbet, som netop er $R_1||R_2$, skal som hovedregel være meget større end udgangsimpedansen i den kreds den belaster. En tommelfingerregel siger at 10 gange større er tilstrækkeligt til at være $"$meget større$"$ hvormed parallelkoblingen skal være over eller lig med 22 k\ohm. 
Belastningen, $R_L$, for det første trin bliver indgangsimpedansen i det andet. Indgangsimpedansen i det andet trin bliver $R_3||R_4$ og kan heller ikke beregnes. Da $R_C$ i det første trin ikke kendes endnu vælges indgangsimpedansen i det andet til at være den samme som i det første, altså 22 k\ohm. 
$V_{R_C}$ er defineret som værende $V_{CC} - V_{\mathrm{CE,sat}} - V_{R_E} - V_{\mathrm{o,peak}}$, hvor $V_{CC}$ vælges til 15 V så der sikres at der er plads til det ønskede spændingsudsving, $V_{R_E}$ vælges til 3 V og $V_{\mathrm{CE,sat}}$ aflæses i databladet for BC547b til 0,2 V ved en collectorstrøm på 1 mA. Der antages at collectorstrømmen ikke bliver større end 1 mA. Ligeledes aflæses $\beta$ til 250 i databladet. $V_{\mathrm{o,peak}}$ er peakspændingen på udgangen. Dermed bliver peakspændingen en faktor 10 højere end mikrofonens output peakspænding. $V_{\mathrm{o,peak}}$ bliver derfor 316 mV. $R_C1$ beregnes hermed i ligning \ref{eq:rcforsteberegning}.

\begin{equation}
R_{\mathrm{C1}} = \sqrt{\frac{22~k\ohm || 2,2~k\ohm \cdot (15~V - 0,2~V - 3~V - 0,316~mV)}{250 \cdot 26 \cdot 10^{-3}}}=9,25~k\ohm
\label{eq:rcforsteberegning}
\end{equation}

$R_{E1}$ bestemmes i ligning \ref{eq:beregningre1} under antagelse at $i_e = i_c$. 

\begin{equation}
i_c=\frac{V_{R_C}}{R_C}
\end{equation}
\begin{equation}
R_{E1}=\frac{V_{R_{E1}}}{\frac{V_{R_{C1}}}{R_{C1}}}  \Rightarrow R_{E1}=\frac{3~V}{\frac{11,5~V}{9,25~k\ohm}}=2,42~k\ohm
\label{eq:beregningre1}
\end{equation}





\subsection*{Design af andet trin}



