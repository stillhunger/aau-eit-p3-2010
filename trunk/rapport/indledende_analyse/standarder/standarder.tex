\section{Standarder}
I dette afsnit opstilles der krav til HiFi-forstærkeren. Disse krav vil tage udgangspunkt i gældende standarder fra International Electrotechnical Commitee (IEC) og Deutsches Institut f\"{u}r Normung (DIN). Målet med standarder er at opstille nogle normer for hvad produkter skal leve op til, dette gøres for at standardisere markedet sådan at de produkter fra forskellige virksomheder kan arbejde sammen og ikke kun virker med produkter fra samme virksomhed. Kravene opstillet i standarderne er ikke lovkrav, men derimod retningslinier. Det er dog i de færreste at det ikke er bedst at overholde standarderne.
\newline
I dette projekt er standarderne brugt til at fremsætte nogle minimumskrav for hvad en HiFi-forstærker skal kunne levere. Standarderne gruppen har valgt at overholde er som følger.


\begin{itemize}              
\item \textbf{IEC581 Part 6 - Amplifiers (1979)}\fixme{Kilde til IEC581-6}
\end{itemize} 
Hele standarden IEC581 har titlen $"$High fidelity audio equipment and systems; Minimum performance requirements$"$. Del 6 af standarden er taget med i rapporten fordi den opstiller generele minimumskrav til hvad en HiFi-forstærker skal overholde.
\begin{itemize}              
\item \textbf{IEC61938 1 udgave (1997)} \fixme{Kilde til IEC61938}                
\end{itemize} 
Standarden IEC61938 har titlen $"$Audio-, video- og audiovisuelle systemer - Indbyrdes forbindelser og matchende værdier - Foretrukne matchende analoge signalværdier$"$. Standarden er brugt i rapporten til at fremstille krav til kravspecifikationen.
\begin{itemize}                   
\item \textbf{DIN 45500 normen (1973)}\fixme{Kilde til DIN 45500}   
\end{itemize} 
Denne standard er forældet. Standarden er taget med i projektet fordi den stadig opstiller mange minimumskrav til en HiFi-forstærker
\newline
\newline
Tabel \ref{tab:standarder_krav} er en opsamling af de krav der er fundet til kravspecifikation, og i tabellen er angivet fra hvilken standard kravet er kommet.

\begin{table}[h]
\centering
\begin{tabular}{l|l|l}
\hline\hline
Område & Minimumskrav & Standard \\
\hline\hline
Udgangseffekt & Min. 10 W mono $^{[1]}$ & IEC581 \\
\hline
Frekvens omrade & 40 Hz - 16000Hz & IEC581 \\
\hline
THD & Max 1 \% $^{[2]}$ & DIN 45500 \\
& Max 0.7 \% $^{[3]}$ & \\
\hline
Belastningsimpedans & & DIN45500 \\
Højtaler & 4 eller 8 \ohm~$^{[4]}$ & \\
Hovedtelefon & 200 eller 400 \ohm~$^{[4]}$ & \\
\hline\hline
\end{tabular}
\caption{Tabel over minimumskrav fra standarder.}
\label{tab:standarder_krav}
\end{table}

\begin{itemize}
\item[]{[1] 1000 Hz sinus signal i 10 min. Ved en omgivelse temperatur på 35°C}
\item[]{[2] Forforstærker og effektforstærker. Målt i effektbåndbredde 40 - 12500 Hz, med en udgangseffekt på minimum 10 W}
\item[]{[3] Forforstærker eller effektforstærker. Målt i effektbåndbredde 40 - 12500 Hz, med en udgangseffekt på minimum 10 W}
\item[]{[4]Tolerance på 20 \%}
\end{itemize}

Der er i dette afsnit blevet opstillet en række krav fra de undersøgte standarder. Disse data vil blive brugt videre i kravspecifikationen.