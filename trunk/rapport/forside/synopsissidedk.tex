\begin{multicols}{2}
\includegraphics[scale=0.35]{forside/aau.png}

\small{\textbf{Titel:\\}
HiFi-forstærker}

\scriptsize{\textbf{School of Information and Communication Technology\\ Elektronik \& IT\\}
Adresse: Fredrik Bajers Vej 7\\
Telefon: 99 40 86 00\\
URL: esn.aau.dk \\}
\end{multicols}
\begin{multicols}{2}

\small{\textbf{Tema:\\}
Analog og digital elektronik

\textbf{Projektperiode:\\}
EIT3, efteråret 2010

\textbf{Projektgruppe:\\}
311

\textbf{Gruppemedlemmer:\\}
Benjamin Krebs\\
Frederik Juul\\
Jacob Hansen\\
Jesper Knudsen\\
Jonas Hansen\\

\textbf{Vejleder:\\}
Jan H. Mikkelsen

\textbf{Vikarierende vejleder:\\}
Ole Kiel Jensen

\textbf{Sidetal:\\}
\pageref{LastPage}

\textbf{Oplagstal:\\}
7

\textbf{Bilagsantal og art:\\}
1 bilags-CD

\textbf{Afsluttet den:\\}
21/12-2010
\\
\textbf{Synopsis:}
}

\fbox{\begin{minipage}{2.8in}
Der er i dette projekt arbejdet med design og fremstilling af en HiFi-forstærker. Målet med en HiFi-forstærker er at forstærke et lydsignal, med så stor præcision som muligt. Fra standarder er der opstillet en række krav som forsøges at overholde.

HiFi-forstærkeren i dette projekt består af fire moduler: Forforstærker, indgangsvælger, volumenkontrol og effektforstærker. Forforstærkerens funktion er at forstærke et mikrofonsignal til linieniveau, for at det kan sendes ind på niveau med et liniesignal. Dette gør at de to signaler kan behandles éns.
Indgangsvælgeren benyttes til at vælge imellem mikrofonsignalet eller stereosignalet: Intet signal, begge signaler eller de enkelte signaler hver for sig kan vælges. Indgangsvælgeren samler desuden signalerne, og skalérer det endelige signal, så outputtet derfra altid vil være på samme niveau.
Volumenkontrollen dæmper det samlede signals amplitude, til et af brugeren ønsket niveau, for at formindske den endelige volumen.
Effektforstærkeren forstærker signalet, hvilket sørger for at der kan afsættes den krævede effekt i højtaleren.
Der konkluderes, på baggrunden af den endelige test, at på trods af afvigelser fra kravspecifikationen 
\end{minipage}}
\newline
~
\newline
~
\newline
~
\newline
~
\end{multicols}

\textit{\scriptsize{Rapportens indhold er frit tilgængeligt, men offentliggørelse (med kildeangivelse) må kun ske efter aftale med forfatterne.}}