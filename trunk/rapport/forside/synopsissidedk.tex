\begin{multicols}{2}
\includegraphics[scale=0.35]{forside/aau.png}

\small{\textbf{Titel:\\}
HiFi-forstærker}

\scriptsize{\textbf{School of Information and Communication Technology\\ Elektronik \& IT\\}
Adresse: Fredrik Bajers Vej 7\\
Telefon: 99 40 86 00\\
URL: esn.aau.dk \\}
\end{multicols}
\begin{multicols}{2}

\small{
\textbf{Tema:\\}
Analog og digital elektronik

\textbf{Projektperiode:\\}
P3, efteråret 2010

\textbf{Projektgruppe:\\}
311

\textbf{Gruppemedlemmer:\\}
Benjamin Krebs\\
Frederik Juul\\
Jacob Hansen\\
Jesper Knudsen\\
Jonas Hansen\\

\textbf{Vejleder:\\}
Jan H. Mikkelsen

\textbf{Vikarierende vejleder:\\}
Ole Kiel Jensen

\textbf{Sidetal:\\}
\pageref{LastPage}

\textbf{Oplagstal:\\}
7

\textbf{Bilagsantal og art:\\}
1 bilags-CD

\textbf{Afsluttet den:\\}
21/12-2010
\\
\textbf{Synopsis:}
}

\fbox{\begin{minipage}{2.8in}
Der er i dette projekt arbejdet med design og beregning af en HiFi-forstærker. Målet med en HiFi-forstærker er at forstærke et signal, med så stor præcision som muligt. Fra standarder er der opstillet minimumskrav for forstærkning og frekvensgang, samt maximumskrav for forvrængning. 

HiFi-forstærkeren i dette projekt består af 4 moduler: En forforstærker, en indgangsvælger, en volumenkontrol og en effektforstærker. Forforstærkerens funktion er at forstærke et mikrofonsignal til linieniveau, for at det kan sendes ind på niveau med et liniesignal. Dette gør at de to signaler kan behandles éns.
Indgangsvælgeren benyttes til at vælge imellem mikrofonsignalet eller stereosignalet: Intet signal, begge signaler eller de enkelte signaler hver for sig kan vælges. Indgangsvælgeren samler desuden signalerne, og skalérer det endelige signal, så outputtet derfra altid vil være på samme niveau.
Volumenkontrollen dæmper det samlede signals amplitude, for at formindske den endelige volumen.
Effektforstærkeren forstærker først signalet, dernæst sørger den for at levere den mængde strøm som er nødvendig, for at afsætte den krævede effekt i højtaleren.
\end{minipage}}
\newline
\end{multicols}

\textit{\scriptsize{Rapportens indhold er frit tilgængeligt, men offentliggørelse (med kildeangivelse) må kun ske efter aftale med forfatterne.}}