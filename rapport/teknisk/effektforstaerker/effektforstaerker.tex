\chapter{Effektforstærker}
\label{effektforstaerker}
Formålet med en effektforstærker er, som nævnt i kapitel \ref{systemopbygning}, at levere en strømforstærkning der gør det muligt at afsætte den ønskede effekt i belastningsmodstanden. Der er i denne rapport valgt at designe en effektforstærker der outputter mere end 20 W i 8 \ohm~\ref{valg_udgangseffekt} . Derudover er der i afsnit \ref{IEC581} bestemt at signalet der kommer ind ikke må forvrænges mere end 0.5 \% .   

 I projektet er der valgt at konstruerer en klasse AB forstærker , hvilket opsætter et krav om at nyttevirkningen i effektforstærkeren bliver højere end 25 \% \ref{klasse_ab}. Der er også valgt at der skal være en kortslutningssikring i effektforstærkeren \ref{valg_kortslutningssikring}.

 Indgangssignalet til effektforstærkeren er fastlagt af standard \fixme{Kilde: IEC61938} til at være mellem 0.2 V til 2 V peak. til sidst er det valgt at effektforstærkeren skal være mono \ref{valg_udgangssignaltype} . Alle kravene der er stillet til effektforstærkeren er opstillet i tabel \ref{tab:krav_effektforstaerker}.

\begin{table}[h]
\centering
\begin{tabular}{l|r}
\hline\hline
Område & Krav \\
\hline\hline
Klasse & AB \\[4pt]
Nyttevirkning & > 25 \%  \\[4pt]
Forvrængning & < 0,5 \% \\[4pt]
Udgangseffekt & > 20 W \\[4pt]
Belastningsimpedans & 8 \ohm \\[4pt]
Udgangssignaltype & Mono \\[4pt]
Kortslutningsstrøm (peak) & 2,24 A \\
Indgangssignal & 0.2 V - 2 V \\[4pt]
\hline\hline
\end{tabular}
\caption{Krav til effektforstærkeren}
\label{tab:krav_effektforstaerker}
\end{table}







