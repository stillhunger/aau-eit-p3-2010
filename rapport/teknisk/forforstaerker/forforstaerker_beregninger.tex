\section{Designstrategi}
Hvad skal dette afsnit gøre?:
Forklare opbygningen af hardwaren og hvorfor lige netop CE med Re er valgt
Forklare hvad der skal tages hensyn til og hvordan det vil blive gjort.
Altså skal der forklares noget tilbagekoblingshalløj og hvordan CE med Re fungerer

\begin{equation}
20 Pa \cdot 5 \frac{mV}{Pa} = 100 mV_{Peak}
\label{eq:mikrofonoutput}
\end{equation}

\subsection*{Kredsløb til mikrofon}

Mikrofonen som benyttes behøver skal, i følge databladet, have en supply spænding fra 1,5 til 10 V og en strøm på 0,5 mA. Desuden skal koblingskondensatorens størrelse være mellem 0,1-4,7$\mu$F. Kredsløbet til mikrofon vises på figur \ref{fig:mikrofonkreds}. 

\begin{figure}[h]
\centering
\includegraphics[scale=.6]{teknisk/forforstaerker/mikrofonkreds.png}
\caption{Supply kredsløb til mikrofonen}
\label{fig:mikrofonkreds}
\end{figure}

Spændingsfaldet over $R_1$ varierer sammen med mikrofonens output. Da strømmen gennem $R_1$ er lavest når spændingsfaldet er på sit minimum skal modstanden dimensioneres efter dette. Dermed bliver størrelse på $R_1$ i følge Ohms lov:

\begin{equation}
R_1 =  \frac{V_{R_1,min}}{I} = \frac{5-315 \cdot 10^{-3}}{0,5 \cdot 10^{-3}} = 9,37 k\Omega
\end{equation}

Koblingskondensatorens, C1, størrelse vil være afhængig af den modstand den kigger ind i og vil derfor blive dimensioneret i sammenhæng med forforstærkerkredsløbet \fixme{ref til forforstærkerkredsløbsdesign. Laaaangt ord.}

\subsection*{Kredsløbsdesign}

Hvad skal dette afsnit gøre?:
Forklare hvilke valg der er blevet truffet i designprocessen og hvorfor.



