\section{Klasser}
\label{klasser}
En HiFi-forstærkers udgangstrin kan designes på forskellige måder alt efter hvilken funktionalitet der ønskes. De forskellige designs er opdelt i klasser. Klasserne er bestemt ud fra en karakteristik og ikke ud fra en bestemt opkobling af kredsløbet. Karakteristika, som er vigtige at tage i betragtning for udgangstrinnet i en HiFi-forstærker er virkningsgrad, strømvinkel og forvrængning. Virkningsgrad er givet ved hvor stor en procentdel af den totale effekt leveret af forsyning, der bliver afsat i loaden, i dette tilfælde højtaleren.
I dette afsnit vil der blive gjort rede for klasse A, B og AB samt forklaret hvilke fordele og ulemper der er med dem. Redegørelsen vil tage udgangspunkt i ovenstående karakteristika samt demonstrere en mulig opbygning af trinnet.
Der vil, på baggrund af dette afsnit, blive valgt en endelig udgangstrinsklasse til dette projekts HiFi-forstærker hvilket vil blive, et krav i kravspecifikationen.

\subsection{Klasse A}

Et klasse A udgangstrin har en strømkarakteristik på udgangen, som vist på figur \ref{fig:klassea} med en sinustone, som indgangssignal. 

\begin{figure}[ht]
\begin{minipage}[b]{0.5\linewidth}
\centering
\includegraphics[scale=.35]{indledende_analyse/klasser/klassea.png}
\caption{Klasse A $i_c$ karakteristik}
\label{fig:klassea}
\end{minipage}
\hspace{0.5cm}
\begin{minipage}[b]{0.5\linewidth}
\centering
\includegraphics[scale=.35]{indledende_analyse/klasser/classa.png}
\caption{Klasse A forstærker kredsløb}
\label{fig:classa}
\end{minipage}
\end{figure}


Et klasse A trin har en strømvinkel på udgangen på 360°. Dette viser sig nyttigt i det at indgangssignalet er repræsenteret på udgangen i sin komplette form, hvilket giver en høj transparenthed, altså en lav forvrængning.
Et klasse A trins udgangsstrøm er centreret om en DC-strøm, hvilket betyder at der afsættes effekt i trinnet i hviletilstand. Dette gør at den maksimale teoretiske virkningsgrad kun er 25\%.

Et klasse A udgangstrin kan opbygges af to NPN transistorer, Q1 og Q2, i en emitterfølgerkobling, som vist på figur \ref{fig:classa}. En konstant strøm løber gennem Q2, da $v_{BE2}$ er konstant. Inputsignalet kommer ind på Q1's base og styrer således strømmen der kan løbe gennem Q1 og loadmodstanden. 



\subsection{Klasse B}

Et klasse B udgangstrin har en strømkarakteristik på udgangen, som vist på figur \ref{fig:klasseb} med en sinustone, som indgangssignal. 

\begin{figure}[ht]
\begin{minipage}[b]{0.5\linewidth}
\centering
\includegraphics[scale=.35]{indledende_analyse/klasser/klasseb.png}
\caption{Klasse B $i_c$ karakteristik}
\label{fig:klasseb}
\end{minipage}
\hspace{0.5cm}
\begin{minipage}[b]{0.5\linewidth}
\centering
\includegraphics[scale=.25]{indledende_analyse/klasser/klassebproblem.png}
\caption{Klasse B trin med crossoverdistortion}
\label{fig:classbproblem}
\end{minipage}
\end{figure}

Et klasse B trin overfører kun en halv periode af indgangssignalet til udgangen, altså er strømvinklen 180°. For at kunne gengive et udgangssignal similært til indgangssignalet er det derfor nødvendigt at sammensætte to klasse B trin således at det ene tager sig af den positive halvperiode og den anden den negative. Dette giver anledning til et fænomen kaldet crossoverdistortion. Dette fænomen optræder i dette tilfælde i overgangen fra den positive halvperiode til den negative og skyldes diodekarakteristikken i transistorernes base-emitter. Crossoverdistortion for et klasse B trin er illustreret på figur \ref{fig:classbproblem}.
Et klasse B trin har en maksimal nyttevirkning på 50\%.

I eksemplet er klasse B trinnet opbygget af to transistorer, en NPN (Q1) og en PNP (Q2), som vist på figur \ref{fig:classb}. Når input spændingen overstiger ca. 0,6 V vil Q1 begynde at lede strøm til loadmodstanden mens Q2 er lukket. Kommer input spændingen under -0,6 V vil Q2 lede, men da Q2 er en PNP vil den trække strøm mod -Vcc hvormed der trækkes strøm fra loadmodstanden. Når Q2 leder er Q1 lukket. 

\begin{figure}[h]
\centering
\includegraphics[scale=.35]{indledende_analyse/klasser/classb.png}
\caption{Klasse B forstærker kredsløb}
\label{fig:classb}
\end{figure}

\subsection{Klasse AB}

Et klasse AB udgangstrin har en strømkarakteristik på udgangen, som vist på figur \ref{fig:klasseab} med en sinustone, som indgangssignal. 

\begin{figure}[ht]
\begin{minipage}[b]{0.5\linewidth}
\centering
\includegraphics[scale=.35]{indledende_analyse/klasser/klasseab.png}
\caption{Klasse AB $i_c$ karakteristik}
\label{fig:klasseab}
\end{minipage}
\hspace{0.5cm}
\begin{minipage}[b]{0.5\linewidth}
\centering
\includegraphics[scale=.35]{indledende_analyse/klasser/classab.png}
\caption{Klasse AB forstærker kredsløb}
\label{fig:classab}
\end{minipage}
\end{figure}


Dette trin har en strømvinkel på mellem 180 ° og 360 °. Dette bevirker at, hvis man bruger samme teknik, som ved et klasse B trin til at få en hel sinusperiode på udgangen, vil de to signaler overlappe i overgangsperioden. Dette medvirker til at crossoverdistortion, som forklaret for klasse B trinnet, elemineres. Dermed bliver forvrængningen for et klasse AB mindre end for et klasse B.
Et klasse AB trin har en maksimal nyttevirkning på 75 \%.

Der tages i eksemplet på et klasse AB trin på figur \ref{fig:classab} udgangspunkt i klasse B trinnet på figur \ref{fig:classb}, med den forskel at potentialet på Q1 og Q2's base er hævet til saturationspændingen når signalspændningen er 0 V. Det er denne forskel, som eleminerer crossoverdistortion.


\subsection{Delkonklusion}

Der vælges at arbejde videre med et klasse AB udgangstrin \fixme{ryk op i AB}
